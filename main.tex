\documentclass[10pt]{beamer}

\usepackage[utf8]{inputenc}
\usepackage{amsmath,amssymb,amsfonts,amsthm,mathrsfs}
\usepackage{mathtools}
\usepackage{hyperref}

\usepackage{verbatim}
\usepackage{graphicx}
\usepackage{scalefnt}
\usepackage{algorithmic}
\usepackage{tikz}
\usepackage{caption}
\usepackage{subcaption}
\newcommand{\inner}[2]{\langle{#1},{#2}\rangle}
\usepackage{multicol}
\setlength{\columnseprule}{0.4pt}
\DeclareMathOperator*{\aff}{aff}
\DeclareMathOperator*{\conv}{conv}
\DeclareMathOperator*{\cone}{cone}
\theoremstyle{definition}
\usepackage{bbm}
\usepackage{lmodern} %for institute name
\usepackage{hyperref}
\usepackage{subfig}
\setcounter{MaxMatrixCols}{20}
\newtheorem{thm}{Theorem}
\newtheorem{prop}{Proposition}
\newtheorem{rem}{Remark}
\newtheorem{defn}{Definition}
\newtheorem{lem}{Lemma}

\everymath{\displaystyle}
\title{Kalman Filters and Localization}

% \institute{KTH Royal Institute of Technology}
\date{\today}

\usetheme{Copenhagen}
\usecolortheme{rose}

\let\olditem\item
\renewcommand{\item}{\setlength{\itemsep}{\fill}\olditem}

\begin{document}
\frame{\titlepage}

\begin{frame}{Localization}
\textbf{Common Sensors}
\begin{itemize}
    \item GPS: Uses time-of-flight to several satellites to estimate x/y/z position
    \item Gyroscope: Electromechanical device for determining the direction of gravity
    \item Accelerometer: Measures the angular acceleration
    \item Magnetometer: Uses a magnet to determine magnetic North
    \item IMU: Integrates gyroscope, accelerometer, magnetometer to determine location. 
\end{itemize}
The actual IMU and GPS algorithms are outside the scope of this presentation.
\end{frame}

\begin{frame}{Kalman Filter}
    Many techniques exist for filtering the data. The provided MATLAB code uses the standard kalman filter.
    
    $\bar{u}_k = u_k - \delta u_k + w_k$
    
    where $u_k$ is a signal, $w_k$ is an additive noise matrix, and $\delta u_k$ is a slowly varying measurement bias:
    
    $ \delta u_k = \delta u_{k-1} + w_k^*$

    and $w_k^*$ is an additive noise matrix with a specific covariance matrix, derived from the data. Each datapoint is then determined by another iterative function:
    $ u_{k+1} = f(u_k)$
    $$ u_k = u_{k-1} - \delta(\delta u_{k-2} + w_k^*) + w_k $$
    Our gyroscope has a known error of .01 degrees latitude per second, showing a cumulative growth in the next slide along the y and z directions.   
\end{frame}

% TASK 1 Use the functions errorgrowth.m and Nav eq.m to evaluate how
% the position error grows with time. Assume that the navigation system is sta-
% tionary; the initial position and velocity is zero; the initial roll, pitch, and yaw
% are zero; the accelerometer measurements are error free; and the gyroscope
% measurements are error free except from a bias in the x-axis gyroscope with
% a magnitude of 0.01 ◦/s. Explain the observed behavior: In what directions
% do you get a position error? Find an approximative formula for the error
% behavior. Where is error largest, Stockholm or Lund?

\begin{frame}{Task 1: Error Growth}
\begin{figure}
    \includegraphics[width=.6\textwidth]{images/error_growth.png}
    \caption{Error Growth over time for unaided IMU.}
    \label{fig:my_label}
    We can see that, as expected, the error rate grows over time without bound.
\end{figure}
\end{frame}

\begin{frame}{Error Bias}
    This error arises from the uncertainty of the accelerometer since the acceleration, $a$ in the $x,z$ directions is initialized as:
    $$
    a_x = 0, a_z = -g
    $$
    $$
    \delta a = 
    \begin{bmatrix}
        \delta a_x \\
        \delta a_z \\
    \end{bmatrix} = 
    \begin{bmatrix} 
        -g \delta a \\
        0           \\
    \end{bmatrix}
    \
    $$
    Integrating over the iterative function from the previous, we know that
    $$
    \delta a(t) = \delta u_0 + \int_0^t err_0 \cdot dt
    $$
    where the $err_0$ is the instantaneous error. There should be no signicant difference in bias between Lund and Stockholm.
\end{frame}
% TASK 2 Familiarize yourself with the Matlab code that implements the
% GNSS-aided INS and execute the script main.m. Modify the code to simulate
% a GNSS-receiver outage from 200 seconds and onward. Plot the estimated car
% position ˆxk (output data.x h). Also plot the difference between estimated
% car position and the GPS position without outage. Experiment with varying
% settings of filter parameters in get settings.m and try to improve the filter
% performance during GPS outage.
\begin{frame}{Task2 : Location}
\begin{figure}

    \subfloat[ With Outage RMS Error = 187.37]{\includegraphics[width=.5\textwidth]{images/outage.png}}
    \subfloat[ Without Outage RMS Error = 1.86]{\includegraphics[width=.5\textwidth]{images/no_outage.png}}
    \caption{Location over time for aided IMU.  The error is too small for these plots to be of much use, even when zoomed in.}
    
\end{figure}
\begin{figure}
    
\end{figure}

\end{frame}

\begin{frame}{Task2 : Difference}
\begin{figure}

    \subfloat[ With Outage RMS Error = 1.65]{\includegraphics[width=.5\textwidth]{images/outage_location.png}}
    \subfloat[ Without Outage RMS Error = 1.56]{\includegraphics[width=.5\textwidth]{images/nooutage_location.png}}
    \caption{Error over time for aided IMU. Using IMU + Accelerometer data.}
    
\end{figure}
\begin{figure}
    
\end{figure}

\end{frame}

%  TASK 3 Implement support for non-holonomic motion constraints
% and speedometer measurements into the GNSS-aided INS framework.
% Some hints where changes are needed are given in the code. Turn
% off speedometer measurements and run with only the motion constraints
% (settings.speed aiding=’off’;settings.non holonomic=’on’ ). Tune
% the filter by changing the magnitude of filter parameters, for example R(3),
% until you get a decent performance and plot the position estimates and po-
% sition error, with GNSS outage. You should see a substantial improvement.

\begin{frame}{Task3 : Non- Holonomic}
\begin{figure}

    \subfloat[X, Y Error, RMS Error = 1.56]{\includegraphics[width=.5\textwidth]{images/nonholo_location.png}}
    \subfloat[Trajectory ]{\includegraphics[width=.5\textwidth]{images/nonholo_trajectory.png}}
    \caption{Error over time for aided IMU. Using IMU + Accelerometer data.}
    
\end{figure}
\end{frame}
% TASK 4 The data included in the GNSSaidedINS.zip folder also include
% measurements from a speedometer. Let the fusion filter also use these mea-
% surements, trim the filter, and then plot the position estimates and position
% error, with GNSS outage. What is the resulting rms error of the position
% trajectory?
\begin{frame}{Task4 : Speed data}
    \begin{figure}

        \subfloat[X, Y Error, RMS Error = 3.9327]{\includegraphics[width=.5\textwidth]{images/speed_location.png}}
        \subfloat[Trajectory ]{\includegraphics[width=.5\textwidth]{images/speed_trajectory.png}}
        \caption{Error over time for aided IMU. Using IMU + Accelerometer data.}
    
\end{figure}
\end{frame}

\end{document}